% about KAOS/SysML by Regine & Christophe

The SysML/KAOS meta-model is an extension of the SysML requirements model with concepts of the KAOS goal model [12]. Several models exist to represent goal-oriented requirements such as i* [14], GBRAM [17]. The choice of KAOS is motivated by the following reasons. Firstly, it permits the expression of several models (goal, agent, object, behavioural models) and relationships between them. Secondly, KAOS provides a powerful and extensive set of concepts to specify goal models. This allows the design of goal hierarchies with a high level of expressiveness that can be considered at different levels of abstraction.

The SysML/KAOS meta-model allows both functional requirements and non-functional requirements to be modelled. This paper focuses on the concepts related to non-functional requirements. For functional requirements concepts, see [18].

\Myfig{meta} shows non-functional concepts as yellow boxes, the gray boxes represent the SysML concepts. The instantiation of the meta-model allows us to obtain a hierarchy of non-functional requirements in the form of goals. Non-Functional Goals (NFG) are organised in containment hierarchies. A non-functional requirement depicted by a goal is either an abstract NFG or an elementary NFG. An abstract NFG may contain several combinations of sub-goals (abstract or elementary). The relationship Refinement becomes an association class between an abstract NFG and its sub-goals. It can be specialised to represent And/Or goal refinements. A goal that cannot be further refined is an elementary goal. 
For example, the goal Security [fridge input data] is an abstract NFG that can be AND-refined into tree sub-goals: Confidentiality [fridge input data], Integrity [fridge input data] and Availability [fridge input data]. Similarly, the sub-goal Availability [fridge input data] is refined into two sub-goals: Availability [Storing RFID information] and Availability [Sensors data].

At the end of the refinement process, it is necessary to identify and express the various alternative ways to satisfy the elementary goals. For that, we consider the concept of contribution goal (Meta-Class Contribution Goal). A contribution goal captures a possible way to satisfy an elementary goal. The association class Contribution describes the characteristics of the contribution. It provides two properties: ContributionNature and ContributionType. The first one specifies whether the contribution is positive or negative, whereas the second one specifies whether the contribution is direct or indirect. 
A positive (or negative) contribution helps positively (or negatively) to the satisfaction of an elementary goal. 
A direct contribution describes an explicit contribution to the elementary non-functional goal. An indirect contribution describes a kind of contribution that is a direct contribution to a given goal but induces an unexpected contribution to another goal.
Consider for example the elementary goal Confidentiality [fridge input data], a possible solution to meet this goal is to use a code 'PIN'; another solution is to require an additional identifier. These two solutions represent thus direct and positive contribution to this goal. Similarly, having high-end sensors contributes directly and positively to the goal Availability [Sensors data], and may contributes indirectly and positively to Integrity [fridge input data]
Finally, the concept of Impact is used to connect non-functional goals to functional goals. It captures the fact that a contribution goal has an effect on functional goals. As the paper is devoted to non-functional requirements, we do not detail further this concept, see [19] for more details.
